%% %%=================================================================
%% %% <UTF-8>
%% %% 北航学位论文模板使用样例
%% %% 请检查下列文件是否完整.
%% %%-----------
%% %% def/buaa.cls                  : LaTeX宏模板文件
%% %% simfang.ttf               : 仿宋字
%% %% simhei.ttf                : 黑体字
%% %% simkai.ttf                : 楷体字
%% %% simsun.ttc                : 宋体字
%% %% huawenxingkai.ttf         : 华文行楷
%% %% def/GBT7714-2015.bst      : 国标参考文献BibTeX样式文件2015(https://github.com/zepinglee/gbt7714-bibtex-style)
%% %% def/GBT7714-2015-NoWarning.bst      : 国标参考文献BibTeX样式文件2015,在参考文献中不再提示缺失信息
%% %% def/logo-buaa.eps         : 论文封皮北航字样
%% %% tex/*.tex                 : 本模板样例中的独立章节
%% %%-----------
%% %% 请统一使用UTF-8编码.
%% %%=================================================================

%=================================================================
% buaa基于ctexbook模板
% 论文样式参考自二〇二五年九月版《北京航空航天大学研究生学位论文撰写规范》
%======================
% 模板导入:
% \documentclass[thesis,doctype,permission,printtype,ostype,titlelength,subjecttype,<ctexbookoptions>]{buaa}
%======================
% 模板选项:
%======================
% I.1论文类型(thesis)
%--------------------
% a.学术硕士论文(master)[缺省值]
% b.专业硕士论文(professional)
% c.学术博士论文(doctor)
% d.专业博士论文(prodoctor)
%--------------------
% I.2文档类型(doctype)
%--------------------
% a. 文献综述(review)[缺省值]
% b. 开题报告(proposal)
%--------------------
% II.密级(permission)
%--------------------
% a.公开(public)[缺省值]
% b.内部(privacy)
% c.秘密(secret=secret3)
% c.1.秘密3年(secret3)
% c.2.秘密5年(secret5)
% c.3.秘密10年(secret10)
% c.4.秘密永久(secret*)
% d.机密(classified=classified5)
% d.1.机密3年(classified3)
% d.2.机密5年(classified5)
% d.3.机密10年(classified10)
% d.4.机密永久(classified*)
% e.绝密(topsecret=topsecret10)
% e.1.绝密3年(topsecret3)
% e.2.绝密5年(topsecret5)
% e.3.绝密10年(topsecret10)
% e.4.绝密永久(topsecret*)
%--------------------
% III.打印设置(printtype)
%--------------------
% a.图书馆版本(library)[缺省值]
% b.打印版本(print)
%--------------------
% IV.系统类型(ostype)
%--------------------
% a.win(win)[缺省值]
% b.linux (linux)
% c.mac (mac)
%--------------------
% V.标题长度(titlelength) % 如果标题过长导致提名页信息移位,请将该项设置为long
%--------------------
% a.short(short)[缺省值]
% b.long (long)
%--------------------
% V.学科类型(subjecttype) % 理工类选择STEM,社科及文学类选择HSS
%--------------------
% a.理工类(STEM)[缺省值]
% b.社科及文学类 (HSS)
%--------------------
% VI.ctexbook设置选项(<ctexbookoptions>)
%--------------------
% ...
%======================
% 其他说明:
% 1. Mac系统请使用mac选项,并使用XeLaTeX编译。
% 2. 可加入额外ctexbook文档类的选项,其将会被传递给ctexbook。
% 3. 参看示例模板 `main.tex` 及其中插入的各章节 `tex/*.tex` 熟悉模板结构和 $LaTeX$ 语法,撰写论文正文。
% 4. 在写公式时,请不要在公式之后加入额外的空行,避免空行所导致的公式与下一行距离过大的问题。
% 5. 在编译时,请使用xelatex进行编译。
% 6. 在写参考文献时,可以先选用GBT7714-2015.bst来查看缺少哪些关键内容,进行补全,如果存在某些内容找不到的问题,则可以使用GBT7714-2015-NoWarning.bst来不显示这些缺乏信息。
% 7. 在写表时,如果存在上下两个线加粗的需求,可以使用\toprule[], \midrule[] 和 \bottomrule[] 来加粗三线表的三条线。
% 8. 在写论文时,可根据tex文件里的注释信息来修改本文的密级,专硕学硕等内容。
% 9. 如果标题过长导致提名页信息移位,请将标题长度项设置为long。

%=================================================================

% 在此修改 论文类型,密级,打印设置,系统类型,标题长短,学科类型
% 学科类型需填写“STEM”或者“HSS”,该项会影响章节条标题的形式
\documentclass[master,review,public,library,win,short,STEM,AutoFakeBold]{def/buaa}

%\setlength{\abovedisplayskip}{1pt}
%\setlength{\belowdisplayskip}{1pt}
%\setlength{\abovedisplayshortskip}{1pt}
%\setlength{\belowdisplayshortskip}{1pt}

%=================================================================
% 开启/关闭引用编号颜色:参考文献,公式,图,表,算法 等……
\refcolor{off}   % 开启: on; 关闭: off[默认];
% 空白页留字,如果不需要空白页显示任何内容则删去方括号及其中的内容即可
\emptypagewords{[ -- This page is a preset empty page -- ]}

%=================================================================
% buaa模板已内嵌以下LaTeX工具包:
%--------------------
% ifthen, etoolbox, titletoc, remreset,
% geometry, fancyhdr, setspace,
% float, graphicx, subfigure, epstopdf,
% array, enumitem,
% booktabs, longtable, multirow, caption,
% listings, algorithm2e, amsmath, amsthm,
% hyperref, pifont, color, soul,
% ---
% For Win: times
% For Lin: newtxtext, newtxmath
% For Mac: times, fontspec
%--------------------
% 请在此处添加额外工具包>>

%=================================================================
% buaa模板已内嵌以下LaTeX宏:
%--------------------
% \highlight{text} % 黄色高亮
%--------------------
% 请在此处添加自定义宏>>
\usepackage{txfonts}
\usepackage{amsmath}
\usepackage{enumitem}
\usepackage[T1]{fontenc}

%%=================================================================
% 论文题目及副标题-{中文}{英文} 注意:论文题目应严格控制在25个汉字(符)以内 
\Title{北航软件学院开题报告/文献综述~\LaTeX{}模板\BUAAThesis{}}{\LaTeX{} Template of Beihang University Thesis \BUAAThesis{}}
\Subtitle{版本 \BUAAThesisVer{}}{Version \BUAAThesisVer{}}

% 学科大类,对应信息页中“申请学位级别”一项的内容,默认工学
\Branch{工学}

% 英文封面申请学位的类型。
% 学术学位博士学位论文填写“Doctor of Philosophy  in 学科英文名称”。
% 学术学位硕士学位论文填写“Master of Arts in 学科英文名称”(哲学、文学、历史学、法学、教育学、艺术学)或“Master of Science in 学科英文名称”。获得一级学科授权的填写一级学科英文名称,获得二级学科授权的填写二级学科英文名称。
% 专业学位博士学位论文根据实际填写“Doctor of 加专业英文名称”。
% 专业学位硕士学位论文根据实际填写“Master of 加授予学位专业英文名称”。英文名称可参考“中国学位与研究生教育学会”网站信息。
\Degree{Master of Science in Control Science and Engineering}

% 院系-{中文}{英文},专业,研究方向,一级学科,学科方向(学术型学位)
\Department{宇航学院}{School of Astronautics}
\Major{控制科学与工程}
\Feild{控制理论与智能系统}
\Discipline{控制科学与工程}
\Direction{控制理论与智能系统}

% 导师信息-{中文名}{英文名}{职称}
\Tutor{导师姓名}{Tutor}{教授}
\Cotutor{副导师姓名}{Cotutor}{研究员}

% 学生姓名-{中文名}{英文名}
\Author{学生姓名}{Student}
% 学生学号
\StudentID{ID123456}
% 专项计划,如无则删除
\SpecialProg{本行仅专项计划研究生添加}

% 中图分类号
\CLC{TP391.4}

% 时间节点-{年}{月}
\DateStart{2025}{12}
\DateEnd{2027}{6}
\DateSubmit{2025}{12}

%%=================================================================
% 摘要-{中文}{英文}
\Abstract{%
  论文摘要是对论文研究内容的高度概括,应体现论文工作的核心思想。博士学
  位论文的中文摘要一般约800~1200字;硕士学位论文的中文摘要一般约500字。摘
  要内容应涉及本项科研工作的目的和意义、研究思想和方法、研究成果和结论,博
  士学位论文应突出论文的创造性成果,硕士学位论文应突出论文的新见解。应具有
  独立性和自含性,即应是一篇简短但意义完整的文章。论文摘要中不要出现图片、
  图表、表格或其他插图材料。
  
  论文的关键词,是为了文献标引工作从论文中选取出来用以表示全文主题内容
  信息的单词或术语,关键词一般为3~5个,按词条的外延层次排列(外延大的排在
  前面)。每个关键词之间用逗号间隔,最后一个关键词后不缀标点符号。
  
  论文摘要的中文版与英文版文字内容要对应。从中文摘要开始编写页码并采用
  双面印刷。“Keywords”与中文摘要部分的关键词对应,每个关键词之间用逗号间隔。
  }{
  The abstract is a concise summary of the research content of the thesis, reflecting the core ideas of the work. For a doctoral dissertation, the Chinese abstract is typically around 800–1,200 words, while for a master's thesis, it is generally about 500 words. The abstract should address the purpose and significance of the research, the methodology and approach, as well as the key findings and conclusions. Doctoral dissertations should emphasize original contributions, while master's theses should highlight novel insights. The abstract must be self-contained and independent, functioning as a complete yet concise standalone text. Figures, charts, tables, or other illustrative materials should not appear in the abstract.
  
  Keywords are terms or phrases selected from the thesis to represent the main thematic content for indexing purposes. Typically, 3–5 keywords are required, arranged in hierarchical order of scope (with broader terms listed first). Keywords are separated by semicolons, with no punctuation following the last keyword.
  
  The Chinese and English versions of the abstract must align in content. Page numbering begins with the Chinese abstract, and the document should be printed double-sided. The "Keywords" section in the English abstract corresponds to the Chinese version, with terms similarly separated by semicolons.
}
% 关键字-{中文}{英文}
\Keyword{北航,学位论文,博士,硕士,中文,\LaTeX{}模板,\BUAAThesis{}}{BeiHang, Degree thesis, PhD, Master, Chinese, \LaTeX{} template, \BUAAThesis{}}

% 图表目录
\Listfigtab{off} % 启用: on[默认]; 关闭: off;

\begin{document}
%%=================================================================
% 标题级别
%--------------------
% \chapter{1 章}
% \section{1.1 小节}
% \subsection{1.1.1 条}
% \subsubsection{1.1.1.1}
% \paragraph{1.1.1.1.1}
% \subparagraph{1.1.1.1.1.1}
%--------------------
%%=================================================================
% \emptypage:插入不计页码的空白页。如果需要第一章第一页从右页起,可以用这条命令插入一张空白页。

% 说明
% !TeX root = ../main.tex
% 本LaTeX模板的一般使用说明
\chapter{模板简介}

这是北航论文\LaTeX{}模板(\CTeX{}-Based)\BUAAThesis{}。本\LaTeX{}模板为北航研究生学位论文模板,适用于硕博士研究生。本\LaTeX{}模板参考自《北京航空航天大学研究生学位论文撰写规范》,根据2025年9月修订版调整,具体要求请参见北航官网研究生院主页“学位工作政策性文件”《北京航空航天大学研究生学位论文撰写规范》及附件,最终成文格式需参考学院要求及打印方意见。


%-----------------------------
\section{项目结构}

本模板共包含以下文件,请对照解压后的压缩包检查文件是否有缺失。

-BUAAThesis

\quad--def

\quad \quad--GBT7714-2015.bst      // 国标参考文献BibTeX样式文件 

\quad \quad--GBT7714-2015-NoWarning.bst // 取消了对于关键信息缺失的告警

\quad \quad--buaa.cls                  // LaTeX宏模板文件

\quad \quad--simfang.ttf               // 仿宋字

\quad \quad--simhei.ttf                // 黑体字

\quad \quad--simkai.ttf                // 楷体字

\quad \quad--simsun.ttc                // 宋体字

\quad \quad--head-doctor.eps       // 论文封皮学术博士学位论文标题

\quad \quad--head-prodoctor.eps    // 论文封皮专业博士学位论文标题

\quad \quad--head-master.eps       // 论文封皮学术硕士学位论文标题

\quad \quad--head-professional.eps // 论文封皮专业硕士学位论文标题

\quad \quad--logo-buaa.eps         // 论文封皮北航字样

\quad--pic

\quad \quad--logo-buaa.eps         // 论文封皮北航字样

\quad \quad--question\_survey.jpg   // 论文出现问题后可参与的问卷二维码

\quad--tex/*.tex                 // LaTeX模板使用说明中的独立章节

\quad--main.tex              // LaTeX模板使用说明

\quad--ref.bib                   // LaTeX模板中的参考文献Bib文件

\quad-- 输出示例.pdf              // main.tex的编译结果

\subsection{各文件的作用}

./def子文件夹下的内容为学位论文模板格式控制文件,通常同学们无需修改该部分内容。
./pic子文件夹存放的是插图文件,用户可以按章节在该文件夹中新建子文件夹,然后存放论文对应章节插图,这样可以方便管理论文插图。
./tex子文件夹中的文件是输出示例各章节的TeX源码,建议同学们也按照分章节的形式建立并管理自己的论文TeX源码。
./main.tex文件是示例TeX主文件源码,这个文件的作用是定义论文基本格式并组织./tex文件夹中的各章节内容和参考文献。
./ref.bib是管理参考文献的Bib文件,包含一些编写模板说明时用到的参考文献。
./README.md是本模板的项目简介,以及版本更新说明。
如果是对\LaTeX 或者编写代码不熟悉的同学,建议直接在./main.tex、./ref.bib和./tex的基础上撰写自己的学位论文,这样可以降低上手难度,相关命令直接对照各文件已有的代码和编译结果学习其效果。

\section{环境配置}

常见的\LaTeX 写作环境有两种,一种是使用Overleaf的在线环境,另一种是使用TeXLive的本地环境。两种写作环境各有优劣:
\begin{itemize}
    \item 在线环境基本无需配置,本地环境需要较复杂的配置。
    \item 在线环境的免费账户有着严苛的编译时长限制,类似毕业论文这样的长篇文章基本不可能通过编译,需要开通订阅才能解锁编译时长限制。
\end{itemize}

\subsection{Overleaf 环境}

将项目压缩包上传至Overleaf(https://cn.overleaf.com/) 后,修改编译选项为 `XeLaTeX` 即可开始写作。

\subsection{本地编译环境}

编译环境请选择“TeXLive+TeXStudio”方案

\subsubsection{TeXLive安装}

MacOS用户点击\href{https://mirror.ctan.org/systems/mac/mactex/MacTeX.pkg}{\textcolor{blue}{MacTeX}}下载并安装“MacTeX”即可(这是一个包含了“TeXLive”环境的程序)。
Windows 和 Linux 用户可参考以下步骤安装“TeXLive+TeXStudio”:
\begin{enumerate}
    \item 前往\href{https://mirrors.tuna.tsinghua.edu.cn/CTAN/systems/texlive/Images/}{\textcolor{blue}{TeXLive Images - 清华大学开源软件镜像站}}下载“texlive.iso”安装包
    \item 装载“texlive.iso”后,Windows 用户点击“install-tl-windows.exe”启动安装程序,Linux 用户请使用“sudo install-pl”启动安装
    \item 修改安装路径(建议安装在非系统盘),点击“安装”,等待安装过程结束
    \item 在终端输入“tex”,出现版本信息等即表示安装成功
    \item 安装TeXStudio编辑器,修改编译器为“XeLaTeX”
    \item 在TeXStudio中打开“main.tex”即可开始写作
\end{enumerate}

具体的安装配置步骤可参考网上教程。
注意在安装之后,可能需要将TeXLive添加到计算机的环境变量。

\section{宏包使用}

本模板必须的文件包括:

\begin{tabular}{ll}
 \verb|def/buaa.cls |                 & $\triangleright$ LaTeX宏模板文件 \\
 \verb|def/GBT7714-2015.bst|      & $\triangleright$ 国标参考文献BibTeX样式文件 \\
 \verb|def/GBT7714-2015-NoWarning.bst|      & $\triangleright$ 不提示缺失信息的参考文献样式文件 \\
 \verb|def/simfang.ttf|           & $\triangleright$ 仿宋\\
 \verb|def/simhei.ttf|            & $\triangleright$ 黑体\\
 \verb|def/simkai.ttf|            & $\triangleright$ 楷体\\
 \verb|def/simsun.ttc|            & $\triangleright$ 宋体\\
 \verb|def/logo-buaa.eps|         & $\triangleright$ 论文封皮北航字样 \\
 \verb|def/head-master.eps|       & $\triangleright$ 论文封皮学术硕士学位论文标题\\
 \verb|def/head-professional.eps| & $\triangleright$ 论文封皮专业硕士学位论文标题\\
 \verb|def/head-doctor.eps|       & $\triangleright$ 论文封皮学术博士学位论文标题\\
 \verb|def/head-prodoctor.eps|    & $\triangleright$ 论文封皮专业博士学位论文标题\\
 \verb|tex/*.tex|                 & $\triangleright$ 本模板样例中的独立章节\\
\end{tabular}\\

在./tex文件中,通过 \verb|\documentclass[| \verb|<thesis>,| \verb|<permission>,| \verb|<printtype>,| \verb|<ostype>,| \verb|<titlelength>,| \verb|<subjecttype>,| \verb|<ctexbookoptions>| \verb|]{buaa}|载入宏包:
\begin{itemize}[leftmargin=3cm]
  \item[{\tt thesis} $\triangleright$]  论文类型(thesis),可选值:\\
    a) 学术硕士论文(\verb|master|)[缺省值]\\
    b) 专业硕士论文(\verb|professional|)\\
    c) 学术博士论文(\verb|doctor|)\\
    d) 专业博士论文(\verb|prodoctor|)
  \item[{\tt permission} $\triangleright$] 密级(permission),可选值: \\
    a) 公开(\verb|public|)[缺省值]\\
    b) 内部(\verb|privacy|)\\
    c) 秘密(\verb|secret|=\verb|secret3|)\\
    --- c.1) 秘密3年(\verb|secret3|)\\
    --- c.2) 秘密5年(\verb|secret5|)\\
    --- c.3) 秘密10年(\verb|secret10|)\\
    --- c.4) 秘密永久(\verb|secret*|)\\
    d) 机密(\verb|classified|=\verb|classified5|)\\
    --- d.1) 机密3年(\verb|classified3|)\\
    --- d.2) 机密5年(\verb|classified5|)\\
    --- d.3) 机密10年(\verb|classified10|)\\
    --- d.4) 机密永久(\verb|classified*|)\\
    e) 绝密(\verb|topsecret|=\verb|topsecret10|)\\
    --- e.1) 绝密3年(\verb|topsecret3|)\\
    --- e.2) 绝密5年(\verb|topsecret5|)\\
    --- e.3) 绝密10年(\verb|topsecret10|)\\
    --- e.4) 绝密永久(\verb|topsecret*|)
  \item[{\tt printtype} $\triangleright$] 打印设置(printtype),可选值: \\
    a) 图书馆版本,不从奇数页开始(\verb|library|)[缺省值]\\
    b) 打印版本,从奇数页开始,上一部分补足空白页(\verb|print|)
  \item[{\tt ostype} $\triangleright$] 系统类型(ostype),可选值: \\
    a) Windows(\verb|win|)[缺省值]\\
    b) Linux(\verb|linux|)\\
    c) Mac(\verb|mac|)
  \item[{\tt titlelength} $\triangleright$] 标题长短(titlelength),可选值: \\
    a) 短标题(通常二十字以下)(\verb|short|)[缺省值]\\
    b) 长标题(通常二十字及以上)(\verb|long|)
  \item[{\tt subjecttype} $\triangleright$] 学科类型(subjecttype),该选项会影响章节条标题的编号形式,可选值: \\
  	a) 理工类(\verb|STEM|)[缺省值]\\
	b) 社科及文学类(\verb|HSS|)  
  \item[{\tt ctexbookoptions} $\triangleright$] {\tt ctexbook}文档类支持的其他选项: \\
    使用{\tt ctexbookoptions}选项传递{\tt ctexbook}文档类支持的其他选项。
    例如,使用{\tt fontset=founder}选项启用方正字体以避免生僻字乱码的问题\footnote{需要系统安装方正字体。}。
\end{itemize}


\setlength{\hangindent}{4em}
模板已内嵌LaTeX工具包:\\
{\tt ifthen},{\tt etoolbox},{\tt titletoc},{\tt remreset},
{\tt geometry},{\tt fancyhdr},{\tt setspace},{\tt float},
{\tt graphicx},{\tt subfigure},{\tt epstopdf},{\tt array},{\tt enumitem},
{\tt booktabs},{\tt longtable},{\tt multirow},{\tt caption},
{\tt listings},{\tt algorithm2e},{\tt amsmath},{\tt amsthm},
{\tt hyperref},{\tt pifont},{\tt color},{\tt soul};\\
For Windows: {\tt times}, {\tt newtxmath};\\
For Linux: {\tt newtxtext}, {\tt newtxmath};\\
For Mac: {\tt times}, {\tt fontspec}。


模板已内嵌宏:\verb|\highlight{text}|(黄色高亮)。

请统一使用UTF-8编码。



%-----------------------------
\section{选项设置}

模板提供了以下功能可选项,同学们可在论文项目主文件(如./main.tex)中设置。

\begin{itemize}[leftmargin=3cm]
  \item[{\tt  $\backslash$refcolor} $\triangleright$]  开启/关闭引用编号颜色,包括参考文献,公式,图,表,算法等\\
  \texttt{on}:开启\\
  \texttt{off}:关闭 [默认]
  \item[{\tt $\backslash$emptypageword} $\triangleright$]  空白页留字
  \item[{\tt $\backslash$Listfigtab} $\triangleright$]  是否使用图表清单目录\\
  \texttt{on}:开启 [默认]\\
  \texttt{off}:关闭
\end{itemize}

\section{章节撰写}
本模板支持以下标题级别,一般情况下不建议使用三级节和更小级别的标题:

\begin{tabular}{ll}
  \verb|\chapter{章}|              & $\triangleright$ 理工类:第一章;社科及文学类:一、章 \\
  \verb|\chapter*{无章号章}|       & $\triangleright$ 无章号章 \\
  \verb|\chaptera{无章号有目录章}| & $\triangleright$ 无章号有目录章 \\
  \verb|\section{节}|              & $\triangleright$ 理工类:1.1 节;社科及文学类:(一)节\\
  \verb|\subsection{小节}|           & $\triangleright$ 理工类:1.1.1 小节;社科及文学类:1、小节\\
  \verb|\subsubsection{三级节}|         & $\triangleright$ 理工类:1.1.1.1 三级节;社科及文学类:(1)三级节\\
  \verb|\paragraph{段}|             & $\triangleright$ 1.1.1.1.1 段\\
  \verb|\subparagraph{小段}|         & $\triangleright$ 1.1.1.1.1.1 小段\\
  \verb|\summary|                  & $\triangleright$ 总结\\
  \verb|\appendix|                 & $\triangleright$ 附录\\
  \verb|\achievement|              & $\triangleright$ 攻读学位期间取得的成果\\
  \verb|\acknowledgments|          & $\triangleright$ 致谢\\
  \verb|\biography|                & $\triangleright$ 作者简介\\
  \verb|\footnote{脚注}|                & $\triangleright$ \ding{192} 脚注\\
\end{tabular}							
%-----------------------------
\section{注意事项}
\begin{itemize}
  \item[$\triangleright$] \textit{中文斜体}将转换为楷体;
  \item[$\triangleright$] \verb|\label{<text>}|中不能使用中文;
  \item[$\triangleright$] 浮动体与正文之间的距离是弹性的,需要根据内容调整;
  \item[$\triangleright$] 命令符与汉字之间请注意加空格以避免undefined错误;
  \item[$\triangleright$] 模板重定义了脚注命令\verb|\footnote{脚注内容}|,需要注意本模板仅支持单页插入最多10条脚注;\footnote{正文中某句话需要具体注释、且注释内容与正文内容关系不大时可以采用脚注方式。}
\end{itemize}
%-----------------------------
\section{意见及问题反馈}

\indent 请作答该问卷:https://www.wjx.cn/vm/PpalYru.aspx\\

\begin{figure}[!h]
	\centering
	\includegraphics[width=.5\textwidth]{pic/question_survey.jpg}
	\caption{问题反馈问卷二维码}
	\label{fig:survey_ques}
\end{figure}

% 示例
% !TeX root = ../main.tex
% 本LaTeX模板的使用示例
\chapter{模板使用说明}

本章给出了撰写论文时可能用到的\LaTeX 基本命令,同学们可以对照./main.tex源码和“输出示例.pdf”文件各部分的对应内容学习模板各命令的作用。

%==============================
\section{参考文献引用}

\BUAAThesis{} 使用BibTeX处理参考文献,方便使用者管理参考文献条目。
参考文献的具体内容以纯文本形式保存在根目录下的ref.bib文件中,每条参考文献信息都严格按照BibTeX格式写入文件。
大部分文献数据库均支持将参考文献导出为BibTeX格式,使用者只需将导出的文献信息顺序写入ref.bib,并在文中按索引引用即可。
参考文献引用按照参考国标7714和北航学位论文撰写规范执行,如果导出参考文献信息时缺失出版地等项目导致引用内容出现“[出版地不详]”等缺省提示,请使用本模板提供的“GBT7714-2015-NoWarning.bst”格式文件屏蔽提示信息。
本模板提供了多种引用参考文献命令,通常在正文中使用\verb|\upcite{}|以上标形式引用文献。

%--------------------------------
\subsection{数字标注}
\noindent
\begin{tabular}{l@{\quad$\Rightarrow$\quad}l}
  \verb|\cite{knuth86a}| & \cite{knuth86a}\\ 
  \verb|\citet{knuth86a}| & \citet{knuth86a}\\
  \verb|\citet[chap.~2]{knuth86a}| & \citet[chap.~2]{knuth86a}\\[0.5ex]
  \verb|\citep{knuth86a}| & \citep{knuth86a}\\
  \verb|\citep[chap.~2]{knuth86a}| & \citep[chap.~2]{knuth86a}\\
  \verb|\citep[see][]{knuth86a}| & \citep[see][]{knuth86a}\\
  \verb|\citep[see][chap.~2]{knuth86a}| & \citep[see][chap.~2]{knuth86a}\\[0.5ex]
  \verb|\citet*{knuth86a}| & \citet*{knuth86a}\\
  \verb|\citep*{knuth86a}| & \citep*{knuth86a}\\
\end{tabular}
\par\noindent
\begin{tabular}{l@{\quad$\Rightarrow$\quad}l}
  \verb|\citet{knuth86a,tlc2}| & \citet{knuth86a,tlc2}\\
  \verb|\citep{knuth86a,tlc2}| & \citep{knuth86a,tlc2}\\
  \verb|\cite{knuth86a,knuth84}| & \cite{knuth86a,knuth84}\\
  \verb|\upcite{knuth86a,knuth84}| & \upcite{knuth86a,knuth84}\\
  \verb|\citet{knuth86a,knuth84}| & \citet{knuth86a,knuth84}\\
  \verb|\citep{knuth86a,knuth84}| & \citep{knuth86a,knuth84}\\
  \verb|\cite{knuth86a,knuth84,tlc2}| & \cite{knuth86a,knuth84,tlc2}\\
\end{tabular}

%--------------------------------
\subsection{数字标注-上标形式}
\noindent
\begin{tabular}{l@{\quad$\Rightarrow$\quad}l}
  \verb|\upcite{knuth86a}| & \upcite{knuth86a}\\
  \verb|\upcite{knuth86a,knuth84,tlc2}| & \upcite{knuth86a,knuth84,tlc2}\\
\end{tabular}
\par\noindent

%--------------------------------
\subsection{著者-出版年制标}
正文中引用参考文献的标注方法可以采用“顺序编码制”,也可以采用“著者-出版年制”。
撰写学位论文时仅选择一种,并全文保持统一。
本模板默认的标注形式为顺序编码制,如果要切换成著者-出版年制,需采用命令\verb|\citestyle{authoryear}|切换。
著者-出版年制标注形式如下:

\citestyle{authoryear}
\noindent
\begin{tabular}{l@{\quad$\Rightarrow$\quad}l}
  \verb|\cite{db}| & \cite{db}\\
  \verb|\citet{knuth86a}| & \citet{knuth86a}\\
  \verb|\citet[chap.~2]{knuth86a}| & \citet[chap.~2]{knuth86a}\\[0.5ex]
  \verb|\citep{knuth86a}| & \citep{knuth86a}\\
  \verb|\citep[chap.~2]{knuth86a}| & \citep[chap.~2]{knuth86a}\\
  \verb|\citep[see][]{knuth86a}| & \citep[see][]{knuth86a}\\
  \verb|\citep[see][chap.~2]{knuth86a}| & \citep[see][chap.~2]{knuth86a}\\[0.5ex]
  \verb|\citet*{knuth86a}| & \citet*{knuth86a}\\
  \verb|\citep*{knuth86a}| & \citep*{knuth86a}\\
\end{tabular}
\par\noindent
\begin{tabular}{l@{\quad$\Rightarrow$\quad}l}
  \verb|\citet{knuth86a,tlc2}| & \citet{knuth86a,tlc2}\\
  \verb|\citep{knuth86a,tlc2}| & \citep{knuth86a,tlc2}\\
  \verb|\cite{knuth86a,knuth84}| & \cite{knuth86a,knuth84}\\
  \verb|\citet{knuth86a,knuth84}| & \citet{knuth86a,knuth84}\\
  \verb|\citep{knuth86a,knuth84}| & \citep{knuth86a,knuth84}\\
\end{tabular}
\citestyle{numbers}

%--------------------------------
\subsection{其他形式的标注}
\noindent
\begin{tabular}{l@{\quad$\Rightarrow$\quad}l}
  \verb|\citealt{tlc2}| & \citealt{tlc2}\\
  \verb|\citealt*{tlc2}| & \citealt*{tlc2}\\
  \verb|\citealp{tlc2}| & \citealp{tlc2}\\
  \verb|\citealp*{tlc2}| & \citealp*{tlc2}\\
  \verb|\citealp{tlc2,knuth86a}| & \citealp{tlc2,knuth86a}\\
  \verb|\citealp[pg.~32]{tlc2}| & \citealp[pg.~32]{tlc2}\\
  \verb|\citenum{tlc2}| & \citenum{tlc2}\\
  \verb|\citetext{priv.\ comm.}| & \citetext{priv.\ comm.}\\
\end{tabular}

\noindent
\begin{tabular}{l@{\quad$\Rightarrow$\quad}l}
  \verb|\citeauthor{tlc2}| & \citeauthor{tlc2}\\
  \verb|\citeauthor*{tlc2}| & \citeauthor*{tlc2}\\
  \verb|\citeyear{tlc2}| & \citeyear{tlc2}\\
  \verb|\citeyearpar{tlc2}| & \citeyearpar{tlc2}\\
\end{tabular}

\section{算法、表格和插图}

根据北航学位论文撰写规范要求,本模板重写了部分图表浮动体环境,但使用方法与官方宏包一致,使用者可查看各宏包的官方文档获取详细使用说明。
需要注意的是图表浮动体与正文之间的距离是弹性的,撰写论文时可以根据内容进行调整。

\subsection{算法环境}

本模板使用 \texttt{algorithm2e} 宏包实现算法环境。下面是四种算法环境示例。

\begin{algorithm}[htp]
  %\SetAlgoLined
  %\SetAlgoVlined
  \caption{A How to (plain).}
  \KwData{this text}
  \KwResult{how to write algorithm with \LaTeX2e{} }

  initialization\;
  \While{not at end of this document}{
    read current\;
    \eIf{understand}{
      go to next section\;
      current section becomes this one\;
    }{
      go back to the beginning of current section\;
    }
  }
\end{algorithm}

\RestyleAlgo{ruled}
\begin{algorithm}[htp]
  \caption{A How to (ruled).}
  \KwData{this text}
  \KwResult{how to write algorithm with \LaTeX2e{} }

  initialization\;
  \While{not at end of this document}{
    read current\;
    \eIf{understand}{
      go to next section\;
      current section becomes this one\;
    }{
      go back to the beginning of current section\;
    }
  }
\end{algorithm}

\RestyleAlgo{boxed}
\begin{algorithm}[htp]
  \caption{A How to (boxed).}
  \KwData{this text}
  \KwResult{how to write algorithm with \LaTeX2e{} }

  initialization\;
  \While{not at end of this document}{
    read current\;
    \eIf{understand}{
      go to next section\;
      current section becomes this one\;
    }{
      go back to the beginning of current section\;
    }
  }
\end{algorithm}

\RestyleAlgo{boxruled}
\begin{algorithm}[htp]
  \caption{A How to (boxruled).}
  \KwData{this text}
  \KwResult{how to write algorithm with \LaTeX2e{} }

  initialization\;
  \While{not at end of this document}{
    read current\;
    \eIf{understand}{
      go to next section\;
      current section becomes this one\;
    }{
      go back to the beginning of current section\;
    }
  }
\end{algorithm}
\vspace{5em}
\subsection{三线表}
学位论文中的表格推荐使用三线表形式,如表~\ref{tab:exampletable}。

\begin{table}[!h]
  \centering
  \caption{表的标题}
  \label{tab:exampletable}
  \begin{tabular}{>{\centering\arraybackslash}p{4cm}>{\centering\arraybackslash}p{4cm}}
    \toprule
	操作系统 & TeX 发行版 \\
    \midrule
    所有 & TeX Live \\
    macOS & MacTeX \\
    Windows & MikTeX \\
    \bottomrule
  \end{tabular}
\end{table}

当表题较长时,本模板会自适应换行处理,如表~\ref{tab:example_long_table}。

\begin{table}[!h]
  \centering
  \caption{长表题示例\upcite{zhudaoqian}:考虑到实验中使用到的面内磁场的大小,以及得到的磁矩稳定翻转条件,在计算中使$\alpha$固定,其余参数则与实验中相同}
  \label{tab:example_long_table}
  \begin{tabular}{c c c c c c c}
    \toprule
    \multirow{2}{*}{\textbf{材料体系}} & \multicolumn{6}{c}{\textbf{参数}} \\
    & $t_F$ & $\mu_0H_{K,eff}$ & $M_s(A\cdot m^(-1))$ & $|\Theta_SH|$ & $\iota$ & $\mu_0H_x$ \\ \midrule
   W/CoFeB & 1 nm & 0.29T & $9\times 10^5$ & 0.32 & 0.31 & 24mT \\
   Ta/CoFeB & 1.2 nm & 0.25T & $1\times 10^6$ & 0.03 & 2 & 20mT \\ \bottomrule
  \end{tabular}
\end{table}

\vspace{-1pt}
\subsection{长表格}

超过一页的表格要使用专门的 \texttt{longtable} 环境(表~\ref{tab:longtable})。\\

\begin{longtable}[h]{ccc}
  % 首页表头
  \caption[长表格演示]{长表格演示}
  \label{tab:longtable}\\
  \toprule
  名称  & 说明 & 备注\\
  \midrule
  \endfirsthead
  % 续页表头
  \caption[]{长表格演示(续)} \\
  \toprule
  名称  & 说明 & 备注 \\
  \midrule
  \endhead
  % 首页表尾
  \hline
  \multicolumn{3}{r}{\small 续下页}
  \endfoot
  % 续页表尾
  \bottomrule
  \endlastfoot

  AAAAAAAAAAAA   &   BBBBBBBBBBB   &   CCCCCCCCCCCCCC   \\
  AAAAAAAAAAAA   &   BBBBBBBBBBB   &   CCCCCCCCCCCCCC   \\
  AAAAAAAAAAAA   &   BBBBBBBBBBB   &   CCCCCCCCCCCCCC   \\
  AAAAAAAAAAAA   &   BBBBBBBBBBB   &   CCCCCCCCCCCCCC   \\
  AAAAAAAAAAAA   &   BBBBBBBBBBB   &   CCCCCCCCCCCCCC   \\
  AAAAAAAAAAAA   &   BBBBBBBBBBB   &   CCCCCCCCCCCCCC   \\
  AAAAAAAAAAAA   &   BBBBBBBBBBB   &   CCCCCCCCCCCCCC   \\
  AAAAAAAAAAAA   &   BBBBBBBBBBB   &   CCCCCCCCCCCCCC   \\
  AAAAAAAAAAAA   &   BBBBBBBBBBB   &   CCCCCCCCCCCCCC   \\
  AAAAAAAAAAAA   &   BBBBBBBBBBB   &   CCCCCCCCCCCCCC   \\
  AAAAAAAAAAAA   &   BBBBBBBBBBB   &   CCCCCCCCCCCCCC   \\
  AAAAAAAAAAAA   &   BBBBBBBBBBB   &   CCCCCCCCCCCCCC   \\
  AAAAAAAAAAAA   &   BBBBBBBBBBB   &   CCCCCCCCCCCCCC   \\
  AAAAAAAAAAAA   &   BBBBBBBBBBB   &   CCCCCCCCCCCCCC   \\
  AAAAAAAAAAAA   &   BBBBBBBBBBB   &   CCCCCCCCCCCCCC   \\
  AAAAAAAAAAAA   &   BBBBBBBBBBB   &   CCCCCCCCCCCCCC   \\
  AAAAAAAAAAAA   &   BBBBBBBBBBB   &   CCCCCCCCCCCCCC   \\
  AAAAAAAAAAAA   &   BBBBBBBBBBB   &   CCCCCCCCCCCCCC   \\
  AAAAAAAAAAAA   &   BBBBBBBBBBB   &   CCCCCCCCCCCCCC   \\
  AAAAAAAAAAAA   &   BBBBBBBBBBB   &   CCCCCCCCCCCCCC   \\
  AAAAAAAAAAAA   &   BBBBBBBBBBB   &   CCCCCCCCCCCCCC   \\
  AAAAAAAAAAAA   &   BBBBBBBBBBB   &   CCCCCCCCCCCCCC   \\
  AAAAAAAAAAAA   &   BBBBBBBBBBB   &   CCCCCCCCCCCCCC   \\
  AAAAAAAAAAAA   &   BBBBBBBBBBB   &   CCCCCCCCCCCCCC   \\
  AAAAAAAAAAAA   &   BBBBBBBBBBB   &   CCCCCCCCCCCCCC   \\
  AAAAAAAAAAAA   &   BBBBBBBBBBB   &   CCCCCCCCCCCCCC   \\
  AAAAAAAAAAAA   &   BBBBBBBBBBB   &   CCCCCCCCCCCCCC   \\
  AAAAAAAAAAAA   &   BBBBBBBBBBB   &   CCCCCCCCCCCCCC   \\
  AAAAAAAAAAAA   &   BBBBBBBBBBB   &   CCCCCCCCCCCCCC   \\
  AAAAAAAAAAAA   &   BBBBBBBBBBB   &   CCCCCCCCCCCCCC   \\
  AAAAAAAAAAAA   &   BBBBBBBBBBB   &   CCCCCCCCCCCCCC   \\
  AAAAAAAAAAAA   &   BBBBBBBBBBB   &   CCCCCCCCCCCCCC   \\
  AAAAAAAAAAAA   &   BBBBBBBBBBB   &   CCCCCCCCCCCCCC   \\
  AAAAAAAAAAAA   &   BBBBBBBBBBB   &   CCCCCCCCCCCCCC   \\
  AAAAAAAAAAAA   &   BBBBBBBBBBB   &   CCCCCCCCCCCCCC   \\
  AAAAAAAAAAAA   &   BBBBBBBBBBB   &   CCCCCCCCCCCCCC   \\
  AAAAAAAAAAAA   &   BBBBBBBBBBB   &   CCCCCCCCCCCCCC   \\
  AAAAAAAAAAAA   &   BBBBBBBBBBB   &   CCCCCCCCCCCCCC   \\
  AAAAAAAAAAAA   &   BBBBBBBBBBB   &   CCCCCCCCCCCCCC   \\
  AAAAAAAAAAAA   &   BBBBBBBBBBB   &   CCCCCCCCCCCCCC   \\
  AAAAAAAAAAAA   &   BBBBBBBBBBB   &   CCCCCCCCCCCCCC   \\
  AAAAAAAAAAAA   &   BBBBBBBBBBB   &   CCCCCCCCCCCCCC   \\
  AAAAAAAAAAAA   &   BBBBBBBBBBB   &   CCCCCCCCCCCCCC   \\
  AAAAAAAAAAAA   &   BBBBBBBBBBB   &   CCCCCCCCCCCCCC   \\
  AAAAAAAAAAAA   &   BBBBBBBBBBB   &   CCCCCCCCCCCCCC   \\
  AAAAAAAAAAAA   &   BBBBBBBBBBB   &   CCCCCCCCCCCCCC   \\
  AAAAAAAAAAAA   &   BBBBBBBBBBB   &   CCCCCCCCCCCCCC   \\
  AAAAAAAAAAAA   &   BBBBBBBBBBB   &   CCCCCCCCCCCCCC   \\
  AAAAAAAAAAAA   &   BBBBBBBBBBB   &   CCCCCCCCCCCCCC   \\
  AAAAAAAAAAAA   &   BBBBBBBBBBB   &   CCCCCCCCCCCCCC   \\
  AAAAAAAAAAAA   &   BBBBBBBBBBB   &   CCCCCCCCCCCCCC   \\
  AAAAAAAAAAAA   &   BBBBBBBBBBB   &   CCCCCCCCCCCCCC   \\
  AAAAAAAAAAAA   &   BBBBBBBBBBB   &   CCCCCCCCCCCCCC   \\
  AAAAAAAAAAAA   &   BBBBBBBBBBB   &   CCCCCCCCCCCCCC   \\
\end{longtable}

\subsection{插图}

本模板支持插入pdf、eps、jpg和png等格式的图片。

\begin{figure}[!h]
  \centering
  \includegraphics[width=.5\textwidth]{pic/logo-buaa}
  \caption{测试图片\\第二行题注}
  \label{fig:logo}
\end{figure}

模板使用了subcaption子图宏包,使用者也可自行替换惯用的宏包,一个简单的子图环境如图所示。

\begin{figure}[!h]
	\centering
	\subcaptionbox{子图1}{\includegraphics[width = 0.45\linewidth]{pic/logo-buaa.eps}\vspace{50pt}}
	\hfill	
	\subcaptionbox{子图2}{\includegraphics[width = 0.3\linewidth]{pic/buaa-mark.jpg}}	
	\caption{测试图片}
\end{figure}

\section{数学环境}

\subsection{数学符号}

模板定义了一些正体(upright)的数学符号:
\begin{center}
  \begin{tabular}{>{\centering\arraybackslash}p{4cm}>{\centering \arraybackslash}p{4cm}}
    \toprule
    符号                 & 命令 \\
    \midrule
    常数$\eu$     & \verb|\eu| \\
    复数单位$\iu$ & \verb|\iu| \\
    微分符号$\diff$ & \verb|\diff| \\
    $\argmax$         & \verb|\argmax| \\
    $\argmin$         & \verb|\argmin| \\
    \bottomrule
  \end{tabular}
\end{center}

更多的例子:
\begin{equation}
\eu^{\iu\pi} + 1 = 0
\end{equation}
\begin{equation}
\frac{\diff^2u}{\diff t^2} = \int f(x) \diff x
\end{equation}
\begin{equation}
\argmin_x f(x)
\end{equation}

\subsection{定理、引理和证明}
模板中使用amsmath和amsthm宏包配置了定理、引理和证明等环境。举例如下。
\begin{definition}
  如果函数$f$的积分是可测的且非负的,我们通过以下方式定义其(扩展)\textbf{Lebesgue积分}:
  \begin{equation}
  \int f = \sup_g \int g,
  \end{equation}
  其中,上确界是在所有满足 $0 \leq g \leq f$ 的可测函数 $g$ 上取得的,且 $g$ 是有界的,并且其支撑集具有有限测度。
\end{definition}

\begin{example}
  $\mathbf{R}^d$ 上的可积(或不可积)函数的简单示例如下:
  \begin{equation}
  f_a(x) =
  \begin{cases}
  |x|^{-a} & \text{if } |x| \leq 1,\\
  0 & \text{if } x > 1.
  \end{cases}
  \end{equation}
  \begin{equation}
  F_a(x) = \frac{1}{1 + |x|^a}, \qquad \text{对所有~} x \in \mathbf{R}^d.
  \end{equation}
  那么 $f_a$ 在 $a < d$ 时恰好是可积的,而 $F_a$ 在 $a > d$ 时恰好是可积的。
\end{example}

\begin{lemma}[Fatou]
  假设 $\{f_n\}$ 是一列满足 $f_n \geq 0$ 的可测函数。如果对几乎处处的 $x$ 都有 $\lim_{n \to \infty} f_n(x) = f(x)$,那么
  \begin{equation}
  \int f \leq \liminf_{n \to \infty} \int f_n.
  \end{equation}
\end{lemma}

\begin{remark}
  我们不排除 $\int f = \infty$ 或 $\liminf_{n \to \infty} f_n = \infty$ 的情况。
\end{remark}

\begin{corollary}
  假设 $f$ 是一个非负可测函数,$\{f_n\}$ 是一列非负可测函数,满足对几乎处处的 $x$ 有 $f_n(x) \leq f(x)$ 且 $f_n(x) \to f(x)$。那么
  \begin{equation}
  \lim_{n \to \infty} \int f_n = \int f.
  \end{equation}
\end{corollary}

\begin{proposition}
  假设 $f$ 在 $\mathbf{R}^d$ 上可积。那么对于每一个 $\epsilon > 0$:
  \begin{enumerate}
    \renewcommand{\theenumi}{\roman{enumi}}
    \item 存在一个有限测度的集合 $B$(例如一个球),使得
    \begin{equation}
    \int_{B^c} |f| < \epsilon.
    \end{equation}
    \item 存在一个 $\delta > 0$,使得
    \begin{equation}
    \int_E |f| < \epsilon \qquad \text{每当~} m(E) < \delta.
    \end{equation}
  \end{enumerate}
\end{proposition}

\begin{theorem}
  假设 $\{f_n\}$ 是一列可测函数,满足当 $n$ 趋向无穷时,$f_n(x)$ 几乎处处收敛于 $f(x)$。如果 $|f_n(x)| \leq g(x)$,其中 $g$ 是可积函数,那么
  \begin{equation}
  \int |f_n - f| \to 0 \qquad \text{当~} n \to \infty,
  \end{equation}
  因此
  \begin{equation}
  \int f_n \to \int f \qquad \text{当~} n \to \infty.
  \end{equation}
\end{theorem}

\begin{proof}
  略。
\end{proof}

\subsection{自定义}

\newtheorem*{axiomofchoice}{选择公理}
\begin{axiomofchoice}
  假设 $E$ 是一个集合,${E_\alpha}$ 是 $E$ 的一组非空子集。那么存在一个函数 $\alpha \mapsto x_\alpha$(一个“选择函数”),使得
  \begin{equation}
  x_\alpha \in E_\alpha,\qquad \text{对所有 }\alpha.
  \end{equation}
\end{axiomofchoice}

\newtheorem{observation}{定理}[chapter]
\begin{observation}
  假设一个偏序集 $P$ 具有这样的性质,即每个链在 $P$ 中都有一个上界。那么集合 $P$ 至少包含一个最大元素。
\end{observation}
\begin{proof}[证明]
  略。
\end{proof}

\newtheorem{observationvar2}[observation]{定理2}
\begin{observationvar2}
  假设一个偏序集 $P$ 满足每个链在 $P$ 中都有一个上界。那么集合 $P$ 至少包含一个最大元素。
\end{observationvar2}
\begin{proof}[证明]
  略。
\end{proof}

% \clearpage



% 总结
% \input{tex/chap_summary}

% 参考文献
% 选用参考文献格式

\Bib{def/GBT7714-2015-NoWarning.bst}{ref.bib}

% 附录
% % !TeX root = ../main.tex
% [附录]
\appendix

\chapter{附录说明}

附录是与论文内容密切相关、但编入正文又影响整篇论文编排的条理和逻辑性的资料,是论文主体的补充内容,可根据需要设置。
本模板定义的附录与正文章属于同一等级,支持节、图、表和公式在附录中的特殊编号形式。下列可能需要编入附录的内容:

\begin{enumerate}[label=\arabic*、]
\item 为了整篇论文材料的完整,但编入正文又有损于编排的条理和逻辑性,这一材料包括比正文更为详尽的信息、研究方法和技术更深入的叙述,建议可以阅读的参考文献题录,对了解正文内容有用的补充信息等;
\item 由于篇幅过大或取材于复制品而不便于编入正文的材料;
\item 不便于编入正文的罕见的珍贵或需要特别保密的技术细节和详细方案(这种情况可单列成册);
\item 对一般读者并非必要阅读,但对专业同行有参考价值的资料;
\item 某些重要的原始数据、过长的数学推导、计算程序、框图、结构图、注释、统计表、计算机打印输出文件等。
\end{enumerate}


% 攻读学位期间成果
% % !TeX root = ../main.tex
% [攻读学位期间取得的成果]
\achievement

本模板定义的研究成果页与正文章属于同一标题等级。
攻读博/硕士学位期间取得的研究成果包括“攻读博/硕士学位期间取得的创新成果”和“攻读博/硕士学位期间参与的主要科研工作”两项。

\section*{攻读博/硕士学位期间取得的创新成果}

可包括攻读博/硕士学位期间发表(含录用)的与学位论文相关的学术论文、发明专利、著作、获奖项目、作品等。

\section*{攻读博/硕士学位期间参与的主要科研工作}

可包括攻读博/硕士学位期间参与的与学位论文相关的主要科研项目等,应列明项目名称,项目来源,研制时间,本人承担的主要工作。

若在学期间未取得相关研究成果,在相应标题下的内容写“无”。各种类型研究成果的书写要求如下:

\section*{学术论文}

参照参考文献书写,尚未刊载但已经接到正式录用函的学术论文加括号注明已被xxxx期刊录用。

\section*{发明专利}

参照参考文献书写,处于申请阶段的专利在专利号位置填写专利申请号,并加括号注明是专利申请号。
当专利申请者或所有者为单位时,可以将专利发明人单独注明列在该条目最后。

\section*{著作}

参照参考文献书写,尚未出版但已被出版社决定出版的专著/译著加括号注明出版社名称和预计出版时间。

\section*{获奖项目}

大致按作者.获奖项目名称.获奖时间方式书写。

\section*{作品}

大致按作者.作品名称.创作时间.材料形式.作品尺寸.作品地点.参展信息.是否获奖等信息方式书写。

\section*{研究报告}

公开的研究报告参照参考文献书写。

\section*{其他}

按适当合理的方式书写。

\section*{科研工作}

大致按项目名称.项目来源.研制时间.本人承担的主要工作方式书写。


% 致谢
% \input{tex/chap_acknowledge}

% 作者简介
% \input{tex/chap_biography}

\vspace{5cm}

\end{document}
