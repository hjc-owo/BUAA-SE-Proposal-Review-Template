% !TeX root = ../main.tex
% 本LaTeX模板的一般使用说明
\chapter{模板简介}

这是北航论文\LaTeX{}模板(\CTeX{}-Based)\BUAAThesis{}。本\LaTeX{}模板为北航研究生学位论文模板,适用于硕博士研究生。本\LaTeX{}模板参考自《北京航空航天大学研究生学位论文撰写规范》,根据2025年9月修订版调整,具体要求请参见北航官网研究生院主页“学位工作政策性文件”《北京航空航天大学研究生学位论文撰写规范》及附件,最终成文格式需参考学院要求及打印方意见。


%-----------------------------
\section{项目结构}

本模板共包含以下文件,请对照解压后的压缩包检查文件是否有缺失。

-BUAAThesis

\quad--def

\quad \quad--GBT7714-2015.bst      // 国标参考文献BibTeX样式文件 

\quad \quad--GBT7714-2015-NoWarning.bst // 取消了对于关键信息缺失的告警

\quad \quad--buaa.cls                  // LaTeX宏模板文件

\quad \quad--simfang.ttf               // 仿宋字

\quad \quad--simhei.ttf                // 黑体字

\quad \quad--simkai.ttf                // 楷体字

\quad \quad--simsun.ttc                // 宋体字

\quad \quad--head-doctor.eps       // 论文封皮学术博士学位论文标题

\quad \quad--head-prodoctor.eps    // 论文封皮专业博士学位论文标题

\quad \quad--head-master.eps       // 论文封皮学术硕士学位论文标题

\quad \quad--head-professional.eps // 论文封皮专业硕士学位论文标题

\quad \quad--logo-buaa.eps         // 论文封皮北航字样

\quad--pic

\quad \quad--logo-buaa.eps         // 论文封皮北航字样

\quad \quad--question\_survey.jpg   // 论文出现问题后可参与的问卷二维码

\quad--tex/*.tex                 // LaTeX模板使用说明中的独立章节

\quad--main.tex              // LaTeX模板使用说明

\quad--ref.bib                   // LaTeX模板中的参考文献Bib文件

\quad-- 输出示例.pdf              // main.tex的编译结果

\subsection{各文件的作用}

./def子文件夹下的内容为学位论文模板格式控制文件,通常同学们无需修改该部分内容。
./pic子文件夹存放的是插图文件,用户可以按章节在该文件夹中新建子文件夹,然后存放论文对应章节插图,这样可以方便管理论文插图。
./tex子文件夹中的文件是输出示例各章节的TeX源码,建议同学们也按照分章节的形式建立并管理自己的论文TeX源码。
./main.tex文件是示例TeX主文件源码,这个文件的作用是定义论文基本格式并组织./tex文件夹中的各章节内容和参考文献。
./ref.bib是管理参考文献的Bib文件,包含一些编写模板说明时用到的参考文献。
./README.md是本模板的项目简介,以及版本更新说明。
如果是对\LaTeX 或者编写代码不熟悉的同学,建议直接在./main.tex、./ref.bib和./tex的基础上撰写自己的学位论文,这样可以降低上手难度,相关命令直接对照各文件已有的代码和编译结果学习其效果。

\section{环境配置}

常见的\LaTeX 写作环境有两种,一种是使用Overleaf的在线环境,另一种是使用TeXLive的本地环境。两种写作环境各有优劣:
\begin{itemize}
    \item 在线环境基本无需配置,本地环境需要较复杂的配置。
    \item 在线环境的免费账户有着严苛的编译时长限制,类似毕业论文这样的长篇文章基本不可能通过编译,需要开通订阅才能解锁编译时长限制。
\end{itemize}

\subsection{Overleaf 环境}

将项目压缩包上传至Overleaf(https://cn.overleaf.com/) 后,修改编译选项为 `XeLaTeX` 即可开始写作。

\subsection{本地编译环境}

编译环境请选择“TeXLive+TeXStudio”方案

\subsubsection{TeXLive安装}

MacOS用户点击\href{https://mirror.ctan.org/systems/mac/mactex/MacTeX.pkg}{\textcolor{blue}{MacTeX}}下载并安装“MacTeX”即可(这是一个包含了“TeXLive”环境的程序)。
Windows 和 Linux 用户可参考以下步骤安装“TeXLive+TeXStudio”:
\begin{enumerate}
    \item 前往\href{https://mirrors.tuna.tsinghua.edu.cn/CTAN/systems/texlive/Images/}{\textcolor{blue}{TeXLive Images - 清华大学开源软件镜像站}}下载“texlive.iso”安装包
    \item 装载“texlive.iso”后,Windows 用户点击“install-tl-windows.exe”启动安装程序,Linux 用户请使用“sudo install-pl”启动安装
    \item 修改安装路径(建议安装在非系统盘),点击“安装”,等待安装过程结束
    \item 在终端输入“tex”,出现版本信息等即表示安装成功
    \item 安装TeXStudio编辑器,修改编译器为“XeLaTeX”
    \item 在TeXStudio中打开“main.tex”即可开始写作
\end{enumerate}

具体的安装配置步骤可参考网上教程。
注意在安装之后,可能需要将TeXLive添加到计算机的环境变量。

\section{宏包使用}

本模板必须的文件包括:

\begin{tabular}{ll}
 \verb|def/buaa.cls |                 & $\triangleright$ LaTeX宏模板文件 \\
 \verb|def/GBT7714-2015.bst|      & $\triangleright$ 国标参考文献BibTeX样式文件 \\
 \verb|def/GBT7714-2015-NoWarning.bst|      & $\triangleright$ 不提示缺失信息的参考文献样式文件 \\
 \verb|def/simfang.ttf|           & $\triangleright$ 仿宋\\
 \verb|def/simhei.ttf|            & $\triangleright$ 黑体\\
 \verb|def/simkai.ttf|            & $\triangleright$ 楷体\\
 \verb|def/simsun.ttc|            & $\triangleright$ 宋体\\
 \verb|def/logo-buaa.eps|         & $\triangleright$ 论文封皮北航字样 \\
 \verb|def/head-master.eps|       & $\triangleright$ 论文封皮学术硕士学位论文标题\\
 \verb|def/head-professional.eps| & $\triangleright$ 论文封皮专业硕士学位论文标题\\
 \verb|def/head-doctor.eps|       & $\triangleright$ 论文封皮学术博士学位论文标题\\
 \verb|def/head-prodoctor.eps|    & $\triangleright$ 论文封皮专业博士学位论文标题\\
 \verb|tex/*.tex|                 & $\triangleright$ 本模板样例中的独立章节\\
\end{tabular}\\

在./tex文件中,通过 \verb|\documentclass[| \verb|<thesis>,| \verb|<permission>,| \verb|<printtype>,| \verb|<ostype>,| \verb|<titlelength>,| \verb|<subjecttype>,| \verb|<ctexbookoptions>| \verb|]{buaa}|载入宏包:
\begin{itemize}[leftmargin=3cm]
  \item[{\tt thesis} $\triangleright$]  论文类型(thesis),可选值:\\
    a) 学术硕士论文(\verb|master|)[缺省值]\\
    b) 专业硕士论文(\verb|professional|)\\
    c) 学术博士论文(\verb|doctor|)\\
    d) 专业博士论文(\verb|prodoctor|)
  \item[{\tt permission} $\triangleright$] 密级(permission),可选值: \\
    a) 公开(\verb|public|)[缺省值]\\
    b) 内部(\verb|privacy|)\\
    c) 秘密(\verb|secret|=\verb|secret3|)\\
    --- c.1) 秘密3年(\verb|secret3|)\\
    --- c.2) 秘密5年(\verb|secret5|)\\
    --- c.3) 秘密10年(\verb|secret10|)\\
    --- c.4) 秘密永久(\verb|secret*|)\\
    d) 机密(\verb|classified|=\verb|classified5|)\\
    --- d.1) 机密3年(\verb|classified3|)\\
    --- d.2) 机密5年(\verb|classified5|)\\
    --- d.3) 机密10年(\verb|classified10|)\\
    --- d.4) 机密永久(\verb|classified*|)\\
    e) 绝密(\verb|topsecret|=\verb|topsecret10|)\\
    --- e.1) 绝密3年(\verb|topsecret3|)\\
    --- e.2) 绝密5年(\verb|topsecret5|)\\
    --- e.3) 绝密10年(\verb|topsecret10|)\\
    --- e.4) 绝密永久(\verb|topsecret*|)
  \item[{\tt printtype} $\triangleright$] 打印设置(printtype),可选值: \\
    a) 图书馆版本,不从奇数页开始(\verb|library|)[缺省值]\\
    b) 打印版本,从奇数页开始,上一部分补足空白页(\verb|print|)
  \item[{\tt ostype} $\triangleright$] 系统类型(ostype),可选值: \\
    a) Windows(\verb|win|)[缺省值]\\
    b) Linux(\verb|linux|)\\
    c) Mac(\verb|mac|)
  \item[{\tt titlelength} $\triangleright$] 标题长短(titlelength),可选值: \\
    a) 短标题(通常二十字以下)(\verb|short|)[缺省值]\\
    b) 长标题(通常二十字及以上)(\verb|long|)
  \item[{\tt subjecttype} $\triangleright$] 学科类型(subjecttype),该选项会影响章节条标题的编号形式,可选值: \\
  	a) 理工类(\verb|STEM|)[缺省值]\\
	b) 社科及文学类(\verb|HSS|)  
  \item[{\tt ctexbookoptions} $\triangleright$] {\tt ctexbook}文档类支持的其他选项: \\
    使用{\tt ctexbookoptions}选项传递{\tt ctexbook}文档类支持的其他选项。
    例如,使用{\tt fontset=founder}选项启用方正字体以避免生僻字乱码的问题\footnote{需要系统安装方正字体。}。
\end{itemize}


\setlength{\hangindent}{4em}
模板已内嵌LaTeX工具包:\\
{\tt ifthen},{\tt etoolbox},{\tt titletoc},{\tt remreset},
{\tt geometry},{\tt fancyhdr},{\tt setspace},{\tt float},
{\tt graphicx},{\tt subfigure},{\tt epstopdf},{\tt array},{\tt enumitem},
{\tt booktabs},{\tt longtable},{\tt multirow},{\tt caption},
{\tt listings},{\tt algorithm2e},{\tt amsmath},{\tt amsthm},
{\tt hyperref},{\tt pifont},{\tt color},{\tt soul};\\
For Windows: {\tt times}, {\tt newtxmath};\\
For Linux: {\tt newtxtext}, {\tt newtxmath};\\
For Mac: {\tt times}, {\tt fontspec}。


模板已内嵌宏:\verb|\highlight{text}|(黄色高亮)。

请统一使用UTF-8编码。



%-----------------------------
\section{选项设置}

模板提供了以下功能可选项,同学们可在论文项目主文件(如./main.tex)中设置。

\begin{itemize}[leftmargin=3cm]
  \item[{\tt  $\backslash$refcolor} $\triangleright$]  开启/关闭引用编号颜色,包括参考文献,公式,图,表,算法等\\
  \texttt{on}:开启\\
  \texttt{off}:关闭 [默认]
  \item[{\tt $\backslash$emptypageword} $\triangleright$]  空白页留字
  \item[{\tt $\backslash$Listfigtab} $\triangleright$]  是否使用图表清单目录\\
  \texttt{on}:开启 [默认]\\
  \texttt{off}:关闭
\end{itemize}

\section{章节撰写}
本模板支持以下标题级别,一般情况下不建议使用三级节和更小级别的标题:

\begin{tabular}{ll}
  \verb|\chapter{章}|              & $\triangleright$ 理工类:第一章;社科及文学类:一、章 \\
  \verb|\chapter*{无章号章}|       & $\triangleright$ 无章号章 \\
  \verb|\chaptera{无章号有目录章}| & $\triangleright$ 无章号有目录章 \\
  \verb|\section{节}|              & $\triangleright$ 理工类:1.1 节;社科及文学类:(一)节\\
  \verb|\subsection{小节}|           & $\triangleright$ 理工类:1.1.1 小节;社科及文学类:1、小节\\
  \verb|\subsubsection{三级节}|         & $\triangleright$ 理工类:1.1.1.1 三级节;社科及文学类:(1)三级节\\
  \verb|\paragraph{段}|             & $\triangleright$ 1.1.1.1.1 段\\
  \verb|\subparagraph{小段}|         & $\triangleright$ 1.1.1.1.1.1 小段\\
  \verb|\summary|                  & $\triangleright$ 总结\\
  \verb|\appendix|                 & $\triangleright$ 附录\\
  \verb|\achievement|              & $\triangleright$ 攻读学位期间取得的成果\\
  \verb|\acknowledgments|          & $\triangleright$ 致谢\\
  \verb|\biography|                & $\triangleright$ 作者简介\\
  \verb|\footnote{脚注}|                & $\triangleright$ \ding{192} 脚注\\
\end{tabular}							
%-----------------------------
\section{注意事项}
\begin{itemize}
  \item[$\triangleright$] \textit{中文斜体}将转换为楷体;
  \item[$\triangleright$] \verb|\label{<text>}|中不能使用中文;
  \item[$\triangleright$] 浮动体与正文之间的距离是弹性的,需要根据内容调整;
  \item[$\triangleright$] 命令符与汉字之间请注意加空格以避免undefined错误;
  \item[$\triangleright$] 模板重定义了脚注命令\verb|\footnote{脚注内容}|,需要注意本模板仅支持单页插入最多10条脚注;\footnote{正文中某句话需要具体注释、且注释内容与正文内容关系不大时可以采用脚注方式。}
\end{itemize}
%-----------------------------
\section{意见及问题反馈}

\indent 请作答该问卷:https://www.wjx.cn/vm/PpalYru.aspx\\

\begin{figure}[!h]
	\centering
	\includegraphics[width=.5\textwidth]{pic/question_survey.jpg}
	\caption{问题反馈问卷二维码}
	\label{fig:survey_ques}
\end{figure}