% !TeX root = ../main.tex
% 本LaTeX模板的使用示例
\chapter{模板使用说明}

本章给出了撰写论文时可能用到的\LaTeX 基本命令,同学们可以对照./main.tex源码和“输出示例.pdf”文件各部分的对应内容学习模板各命令的作用。

%==============================
\section{参考文献引用}

\BUAAThesis{} 使用BibTeX处理参考文献,方便使用者管理参考文献条目。
参考文献的具体内容以纯文本形式保存在根目录下的ref.bib文件中,每条参考文献信息都严格按照BibTeX格式写入文件。
大部分文献数据库均支持将参考文献导出为BibTeX格式,使用者只需将导出的文献信息顺序写入ref.bib,并在文中按索引引用即可。
参考文献引用按照参考国标7714和北航学位论文撰写规范执行,如果导出参考文献信息时缺失出版地等项目导致引用内容出现“[出版地不详]”等缺省提示,请使用本模板提供的“GBT7714-2015-NoWarning.bst”格式文件屏蔽提示信息。
本模板提供了多种引用参考文献命令,通常在正文中使用\verb|\upcite{}|以上标形式引用文献。

%--------------------------------
\subsection{数字标注}
\noindent
\begin{tabular}{l@{\quad$\Rightarrow$\quad}l}
  \verb|\cite{knuth86a}| & \cite{knuth86a}\\ 
  \verb|\citet{knuth86a}| & \citet{knuth86a}\\
  \verb|\citet[chap.~2]{knuth86a}| & \citet[chap.~2]{knuth86a}\\[0.5ex]
  \verb|\citep{knuth86a}| & \citep{knuth86a}\\
  \verb|\citep[chap.~2]{knuth86a}| & \citep[chap.~2]{knuth86a}\\
  \verb|\citep[see][]{knuth86a}| & \citep[see][]{knuth86a}\\
  \verb|\citep[see][chap.~2]{knuth86a}| & \citep[see][chap.~2]{knuth86a}\\[0.5ex]
  \verb|\citet*{knuth86a}| & \citet*{knuth86a}\\
  \verb|\citep*{knuth86a}| & \citep*{knuth86a}\\
\end{tabular}
\par\noindent
\begin{tabular}{l@{\quad$\Rightarrow$\quad}l}
  \verb|\citet{knuth86a,tlc2}| & \citet{knuth86a,tlc2}\\
  \verb|\citep{knuth86a,tlc2}| & \citep{knuth86a,tlc2}\\
  \verb|\cite{knuth86a,knuth84}| & \cite{knuth86a,knuth84}\\
  \verb|\upcite{knuth86a,knuth84}| & \upcite{knuth86a,knuth84}\\
  \verb|\citet{knuth86a,knuth84}| & \citet{knuth86a,knuth84}\\
  \verb|\citep{knuth86a,knuth84}| & \citep{knuth86a,knuth84}\\
  \verb|\cite{knuth86a,knuth84,tlc2}| & \cite{knuth86a,knuth84,tlc2}\\
\end{tabular}

%--------------------------------
\subsection{数字标注-上标形式}
\noindent
\begin{tabular}{l@{\quad$\Rightarrow$\quad}l}
  \verb|\upcite{knuth86a}| & \upcite{knuth86a}\\
  \verb|\upcite{knuth86a,knuth84,tlc2}| & \upcite{knuth86a,knuth84,tlc2}\\
\end{tabular}
\par\noindent

%--------------------------------
\subsection{著者-出版年制标}
正文中引用参考文献的标注方法可以采用“顺序编码制”,也可以采用“著者-出版年制”。
撰写学位论文时仅选择一种,并全文保持统一。
本模板默认的标注形式为顺序编码制,如果要切换成著者-出版年制,需采用命令\verb|\citestyle{authoryear}|切换。
著者-出版年制标注形式如下:

\citestyle{authoryear}
\noindent
\begin{tabular}{l@{\quad$\Rightarrow$\quad}l}
  \verb|\cite{db}| & \cite{db}\\
  \verb|\citet{knuth86a}| & \citet{knuth86a}\\
  \verb|\citet[chap.~2]{knuth86a}| & \citet[chap.~2]{knuth86a}\\[0.5ex]
  \verb|\citep{knuth86a}| & \citep{knuth86a}\\
  \verb|\citep[chap.~2]{knuth86a}| & \citep[chap.~2]{knuth86a}\\
  \verb|\citep[see][]{knuth86a}| & \citep[see][]{knuth86a}\\
  \verb|\citep[see][chap.~2]{knuth86a}| & \citep[see][chap.~2]{knuth86a}\\[0.5ex]
  \verb|\citet*{knuth86a}| & \citet*{knuth86a}\\
  \verb|\citep*{knuth86a}| & \citep*{knuth86a}\\
\end{tabular}
\par\noindent
\begin{tabular}{l@{\quad$\Rightarrow$\quad}l}
  \verb|\citet{knuth86a,tlc2}| & \citet{knuth86a,tlc2}\\
  \verb|\citep{knuth86a,tlc2}| & \citep{knuth86a,tlc2}\\
  \verb|\cite{knuth86a,knuth84}| & \cite{knuth86a,knuth84}\\
  \verb|\citet{knuth86a,knuth84}| & \citet{knuth86a,knuth84}\\
  \verb|\citep{knuth86a,knuth84}| & \citep{knuth86a,knuth84}\\
\end{tabular}
\citestyle{numbers}

%--------------------------------
\subsection{其他形式的标注}
\noindent
\begin{tabular}{l@{\quad$\Rightarrow$\quad}l}
  \verb|\citealt{tlc2}| & \citealt{tlc2}\\
  \verb|\citealt*{tlc2}| & \citealt*{tlc2}\\
  \verb|\citealp{tlc2}| & \citealp{tlc2}\\
  \verb|\citealp*{tlc2}| & \citealp*{tlc2}\\
  \verb|\citealp{tlc2,knuth86a}| & \citealp{tlc2,knuth86a}\\
  \verb|\citealp[pg.~32]{tlc2}| & \citealp[pg.~32]{tlc2}\\
  \verb|\citenum{tlc2}| & \citenum{tlc2}\\
  \verb|\citetext{priv.\ comm.}| & \citetext{priv.\ comm.}\\
\end{tabular}

\noindent
\begin{tabular}{l@{\quad$\Rightarrow$\quad}l}
  \verb|\citeauthor{tlc2}| & \citeauthor{tlc2}\\
  \verb|\citeauthor*{tlc2}| & \citeauthor*{tlc2}\\
  \verb|\citeyear{tlc2}| & \citeyear{tlc2}\\
  \verb|\citeyearpar{tlc2}| & \citeyearpar{tlc2}\\
\end{tabular}

\section{算法、表格和插图}

根据北航学位论文撰写规范要求,本模板重写了部分图表浮动体环境,但使用方法与官方宏包一致,使用者可查看各宏包的官方文档获取详细使用说明。
需要注意的是图表浮动体与正文之间的距离是弹性的,撰写论文时可以根据内容进行调整。

\subsection{算法环境}

本模板使用 \texttt{algorithm2e} 宏包实现算法环境。下面是四种算法环境示例。

\begin{algorithm}[htp]
  %\SetAlgoLined
  %\SetAlgoVlined
  \caption{A How to (plain).}
  \KwData{this text}
  \KwResult{how to write algorithm with \LaTeX2e{} }

  initialization\;
  \While{not at end of this document}{
    read current\;
    \eIf{understand}{
      go to next section\;
      current section becomes this one\;
    }{
      go back to the beginning of current section\;
    }
  }
\end{algorithm}

\RestyleAlgo{ruled}
\begin{algorithm}[htp]
  \caption{A How to (ruled).}
  \KwData{this text}
  \KwResult{how to write algorithm with \LaTeX2e{} }

  initialization\;
  \While{not at end of this document}{
    read current\;
    \eIf{understand}{
      go to next section\;
      current section becomes this one\;
    }{
      go back to the beginning of current section\;
    }
  }
\end{algorithm}

\RestyleAlgo{boxed}
\begin{algorithm}[htp]
  \caption{A How to (boxed).}
  \KwData{this text}
  \KwResult{how to write algorithm with \LaTeX2e{} }

  initialization\;
  \While{not at end of this document}{
    read current\;
    \eIf{understand}{
      go to next section\;
      current section becomes this one\;
    }{
      go back to the beginning of current section\;
    }
  }
\end{algorithm}

\RestyleAlgo{boxruled}
\begin{algorithm}[htp]
  \caption{A How to (boxruled).}
  \KwData{this text}
  \KwResult{how to write algorithm with \LaTeX2e{} }

  initialization\;
  \While{not at end of this document}{
    read current\;
    \eIf{understand}{
      go to next section\;
      current section becomes this one\;
    }{
      go back to the beginning of current section\;
    }
  }
\end{algorithm}
\vspace{5em}
\subsection{三线表}
学位论文中的表格推荐使用三线表形式,如表~\ref{tab:exampletable}。

\begin{table}[!h]
  \centering
  \caption{表的标题}
  \label{tab:exampletable}
  \begin{tabular}{>{\centering\arraybackslash}p{4cm}>{\centering\arraybackslash}p{4cm}}
    \toprule
	操作系统 & TeX 发行版 \\
    \midrule
    所有 & TeX Live \\
    macOS & MacTeX \\
    Windows & MikTeX \\
    \bottomrule
  \end{tabular}
\end{table}

当表题较长时,本模板会自适应换行处理,如表~\ref{tab:example_long_table}。

\begin{table}[!h]
  \centering
  \caption{长表题示例\upcite{zhudaoqian}:考虑到实验中使用到的面内磁场的大小,以及得到的磁矩稳定翻转条件,在计算中使$\alpha$固定,其余参数则与实验中相同}
  \label{tab:example_long_table}
  \begin{tabular}{c c c c c c c}
    \toprule
    \multirow{2}{*}{\textbf{材料体系}} & \multicolumn{6}{c}{\textbf{参数}} \\
    & $t_F$ & $\mu_0H_{K,eff}$ & $M_s(A\cdot m^(-1))$ & $|\Theta_SH|$ & $\iota$ & $\mu_0H_x$ \\ \midrule
   W/CoFeB & 1 nm & 0.29T & $9\times 10^5$ & 0.32 & 0.31 & 24mT \\
   Ta/CoFeB & 1.2 nm & 0.25T & $1\times 10^6$ & 0.03 & 2 & 20mT \\ \bottomrule
  \end{tabular}
\end{table}

\vspace{-1pt}
\subsection{长表格}

超过一页的表格要使用专门的 \texttt{longtable} 环境(表~\ref{tab:longtable})。\\

\begin{longtable}[h]{ccc}
  % 首页表头
  \caption[长表格演示]{长表格演示}
  \label{tab:longtable}\\
  \toprule
  名称  & 说明 & 备注\\
  \midrule
  \endfirsthead
  % 续页表头
  \caption[]{长表格演示(续)} \\
  \toprule
  名称  & 说明 & 备注 \\
  \midrule
  \endhead
  % 首页表尾
  \hline
  \multicolumn{3}{r}{\small 续下页}
  \endfoot
  % 续页表尾
  \bottomrule
  \endlastfoot

  AAAAAAAAAAAA   &   BBBBBBBBBBB   &   CCCCCCCCCCCCCC   \\
  AAAAAAAAAAAA   &   BBBBBBBBBBB   &   CCCCCCCCCCCCCC   \\
  AAAAAAAAAAAA   &   BBBBBBBBBBB   &   CCCCCCCCCCCCCC   \\
  AAAAAAAAAAAA   &   BBBBBBBBBBB   &   CCCCCCCCCCCCCC   \\
  AAAAAAAAAAAA   &   BBBBBBBBBBB   &   CCCCCCCCCCCCCC   \\
  AAAAAAAAAAAA   &   BBBBBBBBBBB   &   CCCCCCCCCCCCCC   \\
  AAAAAAAAAAAA   &   BBBBBBBBBBB   &   CCCCCCCCCCCCCC   \\
  AAAAAAAAAAAA   &   BBBBBBBBBBB   &   CCCCCCCCCCCCCC   \\
  AAAAAAAAAAAA   &   BBBBBBBBBBB   &   CCCCCCCCCCCCCC   \\
  AAAAAAAAAAAA   &   BBBBBBBBBBB   &   CCCCCCCCCCCCCC   \\
  AAAAAAAAAAAA   &   BBBBBBBBBBB   &   CCCCCCCCCCCCCC   \\
  AAAAAAAAAAAA   &   BBBBBBBBBBB   &   CCCCCCCCCCCCCC   \\
  AAAAAAAAAAAA   &   BBBBBBBBBBB   &   CCCCCCCCCCCCCC   \\
  AAAAAAAAAAAA   &   BBBBBBBBBBB   &   CCCCCCCCCCCCCC   \\
  AAAAAAAAAAAA   &   BBBBBBBBBBB   &   CCCCCCCCCCCCCC   \\
  AAAAAAAAAAAA   &   BBBBBBBBBBB   &   CCCCCCCCCCCCCC   \\
  AAAAAAAAAAAA   &   BBBBBBBBBBB   &   CCCCCCCCCCCCCC   \\
  AAAAAAAAAAAA   &   BBBBBBBBBBB   &   CCCCCCCCCCCCCC   \\
  AAAAAAAAAAAA   &   BBBBBBBBBBB   &   CCCCCCCCCCCCCC   \\
  AAAAAAAAAAAA   &   BBBBBBBBBBB   &   CCCCCCCCCCCCCC   \\
  AAAAAAAAAAAA   &   BBBBBBBBBBB   &   CCCCCCCCCCCCCC   \\
  AAAAAAAAAAAA   &   BBBBBBBBBBB   &   CCCCCCCCCCCCCC   \\
  AAAAAAAAAAAA   &   BBBBBBBBBBB   &   CCCCCCCCCCCCCC   \\
  AAAAAAAAAAAA   &   BBBBBBBBBBB   &   CCCCCCCCCCCCCC   \\
  AAAAAAAAAAAA   &   BBBBBBBBBBB   &   CCCCCCCCCCCCCC   \\
  AAAAAAAAAAAA   &   BBBBBBBBBBB   &   CCCCCCCCCCCCCC   \\
  AAAAAAAAAAAA   &   BBBBBBBBBBB   &   CCCCCCCCCCCCCC   \\
  AAAAAAAAAAAA   &   BBBBBBBBBBB   &   CCCCCCCCCCCCCC   \\
  AAAAAAAAAAAA   &   BBBBBBBBBBB   &   CCCCCCCCCCCCCC   \\
  AAAAAAAAAAAA   &   BBBBBBBBBBB   &   CCCCCCCCCCCCCC   \\
  AAAAAAAAAAAA   &   BBBBBBBBBBB   &   CCCCCCCCCCCCCC   \\
  AAAAAAAAAAAA   &   BBBBBBBBBBB   &   CCCCCCCCCCCCCC   \\
  AAAAAAAAAAAA   &   BBBBBBBBBBB   &   CCCCCCCCCCCCCC   \\
  AAAAAAAAAAAA   &   BBBBBBBBBBB   &   CCCCCCCCCCCCCC   \\
  AAAAAAAAAAAA   &   BBBBBBBBBBB   &   CCCCCCCCCCCCCC   \\
  AAAAAAAAAAAA   &   BBBBBBBBBBB   &   CCCCCCCCCCCCCC   \\
  AAAAAAAAAAAA   &   BBBBBBBBBBB   &   CCCCCCCCCCCCCC   \\
  AAAAAAAAAAAA   &   BBBBBBBBBBB   &   CCCCCCCCCCCCCC   \\
  AAAAAAAAAAAA   &   BBBBBBBBBBB   &   CCCCCCCCCCCCCC   \\
  AAAAAAAAAAAA   &   BBBBBBBBBBB   &   CCCCCCCCCCCCCC   \\
  AAAAAAAAAAAA   &   BBBBBBBBBBB   &   CCCCCCCCCCCCCC   \\
  AAAAAAAAAAAA   &   BBBBBBBBBBB   &   CCCCCCCCCCCCCC   \\
  AAAAAAAAAAAA   &   BBBBBBBBBBB   &   CCCCCCCCCCCCCC   \\
  AAAAAAAAAAAA   &   BBBBBBBBBBB   &   CCCCCCCCCCCCCC   \\
  AAAAAAAAAAAA   &   BBBBBBBBBBB   &   CCCCCCCCCCCCCC   \\
  AAAAAAAAAAAA   &   BBBBBBBBBBB   &   CCCCCCCCCCCCCC   \\
  AAAAAAAAAAAA   &   BBBBBBBBBBB   &   CCCCCCCCCCCCCC   \\
  AAAAAAAAAAAA   &   BBBBBBBBBBB   &   CCCCCCCCCCCCCC   \\
  AAAAAAAAAAAA   &   BBBBBBBBBBB   &   CCCCCCCCCCCCCC   \\
  AAAAAAAAAAAA   &   BBBBBBBBBBB   &   CCCCCCCCCCCCCC   \\
  AAAAAAAAAAAA   &   BBBBBBBBBBB   &   CCCCCCCCCCCCCC   \\
  AAAAAAAAAAAA   &   BBBBBBBBBBB   &   CCCCCCCCCCCCCC   \\
  AAAAAAAAAAAA   &   BBBBBBBBBBB   &   CCCCCCCCCCCCCC   \\
  AAAAAAAAAAAA   &   BBBBBBBBBBB   &   CCCCCCCCCCCCCC   \\
\end{longtable}

\subsection{插图}

本模板支持插入pdf、eps、jpg和png等格式的图片。

\begin{figure}[!h]
  \centering
  \includegraphics[width=.5\textwidth]{pic/logo-buaa}
  \caption{测试图片\\第二行题注}
  \label{fig:logo}
\end{figure}

模板使用了subcaption子图宏包,使用者也可自行替换惯用的宏包,一个简单的子图环境如图所示。

\begin{figure}[!h]
	\centering
	\subcaptionbox{子图1}{\includegraphics[width = 0.45\linewidth]{pic/logo-buaa.eps}\vspace{50pt}}
	\hfill	
	\subcaptionbox{子图2}{\includegraphics[width = 0.3\linewidth]{pic/buaa-mark.jpg}}	
	\caption{测试图片}
\end{figure}

\section{数学环境}

\subsection{数学符号}

模板定义了一些正体(upright)的数学符号:
\begin{center}
  \begin{tabular}{>{\centering\arraybackslash}p{4cm}>{\centering \arraybackslash}p{4cm}}
    \toprule
    符号                 & 命令 \\
    \midrule
    常数$\eu$     & \verb|\eu| \\
    复数单位$\iu$ & \verb|\iu| \\
    微分符号$\diff$ & \verb|\diff| \\
    $\argmax$         & \verb|\argmax| \\
    $\argmin$         & \verb|\argmin| \\
    \bottomrule
  \end{tabular}
\end{center}

更多的例子:
\begin{equation}
\eu^{\iu\pi} + 1 = 0
\end{equation}
\begin{equation}
\frac{\diff^2u}{\diff t^2} = \int f(x) \diff x
\end{equation}
\begin{equation}
\argmin_x f(x)
\end{equation}

\subsection{定理、引理和证明}
模板中使用amsmath和amsthm宏包配置了定理、引理和证明等环境。举例如下。
\begin{definition}
  如果函数$f$的积分是可测的且非负的,我们通过以下方式定义其(扩展)\textbf{Lebesgue积分}:
  \begin{equation}
  \int f = \sup_g \int g,
  \end{equation}
  其中,上确界是在所有满足 $0 \leq g \leq f$ 的可测函数 $g$ 上取得的,且 $g$ 是有界的,并且其支撑集具有有限测度。
\end{definition}

\begin{example}
  $\mathbf{R}^d$ 上的可积(或不可积)函数的简单示例如下:
  \begin{equation}
  f_a(x) =
  \begin{cases}
  |x|^{-a} & \text{if } |x| \leq 1,\\
  0 & \text{if } x > 1.
  \end{cases}
  \end{equation}
  \begin{equation}
  F_a(x) = \frac{1}{1 + |x|^a}, \qquad \text{对所有~} x \in \mathbf{R}^d.
  \end{equation}
  那么 $f_a$ 在 $a < d$ 时恰好是可积的,而 $F_a$ 在 $a > d$ 时恰好是可积的。
\end{example}

\begin{lemma}[Fatou]
  假设 $\{f_n\}$ 是一列满足 $f_n \geq 0$ 的可测函数。如果对几乎处处的 $x$ 都有 $\lim_{n \to \infty} f_n(x) = f(x)$,那么
  \begin{equation}
  \int f \leq \liminf_{n \to \infty} \int f_n.
  \end{equation}
\end{lemma}

\begin{remark}
  我们不排除 $\int f = \infty$ 或 $\liminf_{n \to \infty} f_n = \infty$ 的情况。
\end{remark}

\begin{corollary}
  假设 $f$ 是一个非负可测函数,$\{f_n\}$ 是一列非负可测函数,满足对几乎处处的 $x$ 有 $f_n(x) \leq f(x)$ 且 $f_n(x) \to f(x)$。那么
  \begin{equation}
  \lim_{n \to \infty} \int f_n = \int f.
  \end{equation}
\end{corollary}

\begin{proposition}
  假设 $f$ 在 $\mathbf{R}^d$ 上可积。那么对于每一个 $\epsilon > 0$:
  \begin{enumerate}
    \renewcommand{\theenumi}{\roman{enumi}}
    \item 存在一个有限测度的集合 $B$(例如一个球),使得
    \begin{equation}
    \int_{B^c} |f| < \epsilon.
    \end{equation}
    \item 存在一个 $\delta > 0$,使得
    \begin{equation}
    \int_E |f| < \epsilon \qquad \text{每当~} m(E) < \delta.
    \end{equation}
  \end{enumerate}
\end{proposition}

\begin{theorem}
  假设 $\{f_n\}$ 是一列可测函数,满足当 $n$ 趋向无穷时,$f_n(x)$ 几乎处处收敛于 $f(x)$。如果 $|f_n(x)| \leq g(x)$,其中 $g$ 是可积函数,那么
  \begin{equation}
  \int |f_n - f| \to 0 \qquad \text{当~} n \to \infty,
  \end{equation}
  因此
  \begin{equation}
  \int f_n \to \int f \qquad \text{当~} n \to \infty.
  \end{equation}
\end{theorem}

\begin{proof}
  略。
\end{proof}

\subsection{自定义}

\newtheorem*{axiomofchoice}{选择公理}
\begin{axiomofchoice}
  假设 $E$ 是一个集合,${E_\alpha}$ 是 $E$ 的一组非空子集。那么存在一个函数 $\alpha \mapsto x_\alpha$(一个“选择函数”),使得
  \begin{equation}
  x_\alpha \in E_\alpha,\qquad \text{对所有 }\alpha.
  \end{equation}
\end{axiomofchoice}

\newtheorem{observation}{定理}[chapter]
\begin{observation}
  假设一个偏序集 $P$ 具有这样的性质,即每个链在 $P$ 中都有一个上界。那么集合 $P$ 至少包含一个最大元素。
\end{observation}
\begin{proof}[证明]
  略。
\end{proof}

\newtheorem{observationvar2}[observation]{定理2}
\begin{observationvar2}
  假设一个偏序集 $P$ 满足每个链在 $P$ 中都有一个上界。那么集合 $P$ 至少包含一个最大元素。
\end{observationvar2}
\begin{proof}[证明]
  略。
\end{proof}

% \clearpage

